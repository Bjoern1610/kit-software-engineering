\section{Planungsphase}

Ziel:
\begin{itemize}
\item Beschreibung des Systems in Worten als \textbf{Lastenheft}
\item \textbf{Durchführbarkeitsuntersuchung}
\end{itemize}
	
\subsection{Lastenheft}
	
\begin{itemize}
\item Zielbestimmung
\item Produkteinsatz
\item Funktionale Anforderungen
\begin{itemize}
\item Beschreibt Funktionen, die das System unterstützen muss, unabhängig von der Implementierung
\item Als Aktionen formuliert: \dq Ersterfassung, Änderung und Kunden\dq
\end{itemize}
\item Produktdaten
\item Nichtfunktionale Anforderungen
\begin{itemize}
\item Beschreiben Eigenschaften des Systems: \dq Reagiert innerhalb von zehn Sekunden\dq
\item Als Einschränkung (constraints) oder Zusicherung (assertions) formuliert
\end{itemize}
\item Systemmodelle
\begin{itemize}
\item Szenarien
\item Anwendungsfälle
\end{itemize}
\item Glossar
\begin{itemize}
\item Begriffslexikon zur einheitlichen Kommunikation mit dem Kunden
\end{itemize}
\end{itemize}
		
\subsection{Durchführbarkeitsuntersuchung}
			
\begin{itemize}
\item Fachliche Durchführbarkeit
\begin{itemize}
\item Fachkräfte genügend qualifiziert?
\end{itemize}
\item Alternative Lösungsvorschläge
\begin{itemize}
\item Open-Source als Teilersatz?
\end{itemize}
\item Personelle Durchführbarkeit
\begin{itemize}
\item Genügend qualifizierte Fachkräfte?
\end{itemize}
\item Risiken
\item Ökonomische Durchführbarkeit
\begin{itemize}
\item Projekt wirtschaftlich? (Aufwands- und Terminschätzung, Wirtschaftlichkeitsrechnung)
\end{itemize}
\item Rechtliche Gesichtspunkte
\begin{itemize}
\item Datenschutz
\item Zertifizierung
\end{itemize}
\end{itemize}