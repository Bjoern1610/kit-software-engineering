\section{Aufwandsschätzung}
		
	\subsection{Schätzmethoden}
			
		\begin{itemize}
			\item \textbf{Analogiemethode}
			\begin{itemize}
				\item Vergleiche die zu schätzenden Entwicklungen mit \textbf{bereits abgeschlossenen Produktentwicklungen} anhand von \textbf{Ähnlichkeitskriterien}
			\end{itemize}
			\item \textbf{\textbf{Basismethoden}}
			\begin{itemize}
				\item \textbf{Relationsmethode}
				\begin{itemize}
					\item Methode, die anhand von \textbf{Faktoren} (Programmiersprache/-erfahrung, Dateiorganisation) vergleicht, wie diese den \textbf{Aufwand beeinflussen}: \textbf{Auf- und Abschläge} mit etwa gleich großem, existierenden Produkt
				\end{itemize}
				\item \textbf{Multiplikatormethode}
				\begin{itemize}
					\item \textbf{Zerlegung in Teilprodukte} mit Zuteilung feststehender Aufwände
									
					$\Rightarrow$ Anzahl Teilprodukte {\Large $\cdot$} Aufwand Kategorie
				\end{itemize}
				\item \textbf{Phasenaufteilung}
				\begin{itemize}
					\item Ermittlung aus abgeschlossenen Entwicklungen werden \textbf{auf einzelne Entwicklungsphasen verteilt} (Kuchendiagramm)
				\end{itemize}
			\end{itemize}
			\item \textbf{COCOMO II}
			\begin{itemize}
				\item {\LARGE $PM = A \cdot (Size)^{1,01 + 0,01 \cdot {\sum_{j=1}^{5} SF_{j}}} \cdot \prod_{i=1}^{17} EM_{i}$}
				\begin{itemize}
					\item $PM$ = Personenmonate
					\item $A$ = Konstante für Kalibrierung des Modells (z.B. LOC)
					\item $Size$ = Geschätzter Umfang der Software in KLOC
					\item $SF_{j}$ = Skalierungsfaktoren
					\item $EM_{i}$ = Multiplikative Kostenfaktoren
				\end{itemize}
			\end{itemize}
		\end{itemize}