\section{Definitionsphase}

	Ziel:
	\begin{itemize}
		\item Erstellung eines \textbf{Pflichtenhefts}, mit Hilfe von \textbf{Objekt}- und \textbf{dynamischen Modellen}
	\end{itemize}
		
		\subsection{Pflichtenheft}
			
			\begin{itemize}
				\item  \textbf{Definiert} das zu erstellende System \textbf{vollständig und exakt}
				\begin{itemize}
					\item \underline{Ohne Nachfragen} implementierbar!
					\item Nicht wie, sondern nur \underline{was zu implementieren} ist
				\end{itemize}
				\item Verfeinerung des Lastenhefts
			\end{itemize}
				
		\subsection{Modellarten}
				
			\begin{itemize}
				\item Funktionales Modell (aus dem Lastenheft)
				\begin{itemize}
					\item Szenarien
					\item Anwendungsfalldiagramme
				\end{itemize}
				\item Objektmodell
				\begin{itemize}
					\item Klassendiagramm
					\item Objektdiagramm
				\end{itemize}
				\item Dynamisches Modell
				\begin{itemize}
					\item Sequenzdiagramm
					\item Zustandsdiagramm
					\item Aktivitätsdiagramm
				\end{itemize}
			\end{itemize}
				
		\newpage
		\subsection{Gliederung}
				
			\begin{itemize}
				\item Zielbestimmung 
				\item Produkeinsatz
				\item \textbf{\textit{Produktumgebung}}
				\item Funktionale Anforderungen
				\item Produktdaten
				\item Nichtfunktionale Anforderungen
				\item \textbf{\textit{Globale Testfälle}}
				\item Systemmodelle
				\begin{itemize}
					\item Szenarien
					\item Anwendungsfälle
					\item \textbf{\textit{Objektmodelle}}
					\item \textbf{\textit{Dynamische Modelle}}
					\item \textbf{\textit{Benutzerschnittstelle - Bildschirmskizze, Navigationspfade}}
				\end{itemize}
				\item Glossar
			\end{itemize}
				
		\subsection{Liskov'sches Substitutionsprinzip}
			
			\begin{itemize}
				\item In einem Programm, in dem U eine Unterklasse von K ist, kann \textbf{jedes Exemplar der Klasse K durch ein Exemplar von U ersetzt werden}, wobei das Programm weiterhin \underline{korrekt funktioniert}
			\end{itemize}
				
		\newpage
		\subsection{Folgerungen aus dem Substitutionsprinzip}
			
			\begin{itemize}
				\item Signaturvererbung
				\begin{itemize}
					\item Eine in der Oberklasse definierte und evtl. implementierte Methode überträgt \textbf{nur ihre Signatur} auf die Unterklasse
				\end{itemize}
				\item Implementierungsvererbung
				\begin{itemize}
					\item Eine in der Oberklasse definierte und implementierte Methode überträgt ihre \textbf{Signatur und ihre Implementierung} auf die Unterklasse
									
					$\Rightarrow$ Implementierungsvererbung setzt Signaturvererbung voraus!
				\end{itemize}
				\item Anpassung geerbter Eigenschaften
				\begin{itemize}
					\item Überladen
					\begin{itemize}
						\item Eine geerbte Methode mit gleichem Namen, aber anderer Signatur wird definiert
					\end{itemize}
					\item Überschreiben
					\begin{itemize}
						\item Eine geerbte, \textbf{dynamische} Methode mit gleichem Namen und gleicher Signatur wird \textbf{neu implementiert}
					\end{itemize}
					\item Verdecken
					\begin{itemize}
						\item Eine geerbte, \textbf{statische} Methode mit gleichem Namen und gleicher Signatur wird \textbf{neu implementiert}
					\end{itemize}
				\end{itemize}
				\newpage
				\item Varianz
				\begin{itemize}
					\item Definition
					\begin{itemize}
						\item Parametermodifikation einer überschriebenen Methode
					\end{itemize}
					\item Invarianz
					\begin{itemize}
						\item Der Parametertyp wird nicht modifiziert
					\end{itemize}
					\item Kovarianz
					\begin{itemize}
						\item Der Parametertyp wird spezialisiert
					\end{itemize}
					\item Kontravarianz
					\begin{itemize}
						\item Der Parametertyp wird allgemeiner
					\end{itemize}
				\end{itemize}
			\end{itemize}
				
			\subsubsection{Varianzen gemäß dem Substitutionsprinzip und in Java}
			
				\begin{center}
					\resizebox{\textwidth}{!}{
					\begin{tabular}{l|c|c|c|c|c|c}
						                              & \multicolumn{3}{c|}{\textbf{Eingabeparameter}} & \multicolumn{3}{c}{\textbf{Ausgabeparameter} } \\
						\hline
						                              & Invarianz                                      & Kovarianz & Kontravarianz & Invarianz & Kovarianz & Kontravarianz \\
						\hline
						\textbf{Substitutionsprinzip} & $\surd$                                        &           & $\surd$       & $\surd$   & $\surd$   & \\
						\hline
						\textbf{Java}                 & $\surd$                                        &           &               & $\surd$   & $\surd$   & \\
					\end{tabular}}
				\end{center}
		
		\subsection{Kapselungsprinzip}
				
			\begin{itemize}
				\item Der \textbf{Zustand} ist zwar nach außen sichtbar, er wird aber \textbf{im Inneren des Objektes verwaltet} (und also nur kontrolliert geändert)
			\end{itemize}
				
		\subsection{Geheimnisprinzip}
			
			\begin{itemize}
				\item Jedes Modul verbirgt eine \textbf{wichtige Entwurfsentscheidung} hinter einer \textbf{wohldefinierten Schnittstelle} die sich bei einer Änderung der Entscheidung \underline{nicht} mit ändert
				\begin{itemize}
					\item \underline{Verborgenes und Unbenutztes} kann ohne Risiko geändert werden
				\end{itemize}
			\end{itemize}
				
		\subsection{Beispiele für Verbergung}
				
			\begin{itemize}
				\item \textbf{Datenstrukturen} (Wahl, Größe und Implementierung und Operationen an diesen)
				\item \textbf{Maschinennahe Details} (Gerätetreiber, Ein- und Ausgabe)
				\item \textbf{Betriebssystemnahe Details} (Ein- und Ausgabeschnittstellen, Dateiformate, Netzwerkprotokolle)
				\item \textbf{Grundsoftware} (Datenbanken, Oberflächenbibliotheken)
				\item \textbf{Benutzungsschnittstellen} (Kommandoschnittstelle, graphische Oberfläche, Gesten-gesteuerte Oberfläche, Web, Sprachsteuerung, Kombinationen davon,..)
				\item \textbf{\underline{Sprache}} (Text von Dialogen, Beschriftungen)
				\item \textbf{Reihenfolge der Verarbeitung}
			\end{itemize}